\documentclass{beamer} 

% Michael Maier, 2014.
% CC-BY-SA 3.0 at

\usepackage[utf8]{inputenc}
\usepackage[ngerman]{babel}

\title{OpenStreetMap - Die freie Weltkarte nutzen} 
\author{Michael Maier \textless Michael.Maier@student.tugraz.at\textgreater} 
\date{13. Jänner 2014} 

\usetheme{Antibes}

\hypersetup{colorlinks=true,urlcolor=blue,linkcolor=white}

%\usebackgroundtemplatei{
%\includegraphics[width=\paperwidth,
%height=0.8\paperheight]{mag_map.png}
%}

\begin{document}

%\maketitle

\begin{frame} 


\begin{figure}
  \centering
  \includegraphics[width=.5\textwidth]{mag_map.png}
\end{figure}

\begin{center}
\Huge{OpenStreetMap\\}
\end{center}

\begin{center}
\Large{\emph{Die freie Weltkarte nutzen}}
\end{center}

\end{frame}


% Zielgruppe
% * Google, Konkurrenten
% * Landes-GISler
% * Firmen

% Wichtig:
% Quality Assurance
% * Es gibt automatische Q/A-Tools
% * kaum Streitfälle - wenn dann Mailinglist, Data Working Group
%    Mapper mittels eingebauten Social Networks direkt kontaktieren
% * 
% tolle Bilder herzeigen!
% * irgendein Zoo
% * 3D
% * Tolle Kartenstile:
%     * OSM-Fr?
%     * stamen watercolor
%     * pistemap
%     * bicycle map
%     * OpenSeaMap
%
% Dokumentation (nenne es doku und nicht Wiki → verwirrent) -OK
% wenn wo doku fehlt, selber mitschreiben?

% Vorteile
% * aktuell!
%   wenn nicht aktuell: OSM Verbessern!
%  - JEDE Karte ist in dem Moment veraltet, wo sie gedruckt wurde...
%  - impressive Number: changesets/day - oder sekunde/minute  -OK
    % changeset 19980639 an  13 Jan 2014 22:59:23
    % 16561091 an 15 Jun 2013 11:10:25
    % = 3419548 in 208d = 16500/d = 685/h = 12/Minute
    % oder 2M nodes/day dazu
%
% * 100e verschiedene Kartenstile 
%   den eigenen leichtgemacht mit Tilemill
% * Datenversionierung! -OK
% * Language-Independent
%   show http://toolserver.org/~osm/locale/ru.html
% * jetzt in 3D!
% switch2osm net vergessen!
% HOT Team nicht vergessen! → zu geschichtliches

% Dienst: Geocoder!
% Dienste: Routing!

% * kurze OSM-Vorstellung, Geschichtliches, Motivation -OK
% * Technogie, Datenmodell, Lizenz
% * OSM Nutzen: Rohdaten, Web-Dienste, Apps

% 3
% Rohdaten
%   link zu jedem item
%   export als svg/osm.org
%   download von planet / shapefiles geofabrik
%   overpass (turbo)
% web-dienste
%   eigene kartenstile machbar
%      Tilemill
%      einige karten vorstellen 
%      Language-Independent
%   Geocoder
%   Notes-Feature
%   Web-Editor

\section{Einleitung}

\begin{frame}{Vorstellung}

  \begin{itemize}
    \item Michael Maier \textless \href{mailto:Michael.Maier@student.tugraz.at}{Michael.Maier@student.tugraz.at}\textgreater
    \item Student an der TU Graz (Telematik)
\vspace{0.3cm}
    \item OpenStreetMap als Hobby seit Juli 2010
    \item Leite den Grazer OSM-Stammtisch seit Mai 2011
\vspace{0.3cm}
    \item Vorträge und Workshops zum Thema OSM seit 2012
    \item Freiberuflich OSM-Aufträge und Consulting
%    \begin{itemize}
%      \item OSM-username: \emph{\href{http://www.openstreetmap.org/user/species}{species}}
%      \item Mapping-area Graz, Leoben on bicycle, motorbike or on public transport
%    \end{itemize}
  \end{itemize}
\end{frame}



% Folien zu
% * kurze OSM-Vorstellung, Geschichtliches, Motivation
%  1. OSM-Vorstellung
  % was ist es
  % wer steckt dahinter?
% Geschichtliches
  % Gegründet ... steve
  % user-wachstum
% Motivation
  % gegründet, weil es keine freien Geodaten gab
  % Wunschtraum: eine DB weltweit

\begin{frame}{Was ist OpenStreetMap}

\begin{itemize}
  \item OpenStreetMap (OSM) ist eine freie Weltkarte nach dem Wiki-Prinzip "`Wikipedia der Karten"'
    \begin{itemize}
      \item \emph{Eigentlich eine Geo-Datenbank}
    \end{itemize}
\pause
  \item Entsteht aus der Arbeit von \textgreater 1,4\,M Hobbykartografen "`\emph{Mapper}"'

 \item Das komplette "`planet file"' ist ca. 427\,GB groß (xml) (Dienstag):
  \begin{itemize}
    \item 2.161.810.973 Nodes
    \item 213.302.643 Ways
    \item 2.326.786 Relations
  \end{itemize}

\end{itemize}


% \begin{center}
% \includegraphics[width=5.5cm]{sotm.jpg}
% \end{center}

\end{frame}

\begin{frame}{Wer steht hinter OpenStreetMap}

  \begin{itemize}
    \item OpenStreetMap Foundation (Server, Rechtliche Vertretung)
      \pause
    \item Mapper ($\sim$20.000 aktiv), meist ohne Geo-Hintergrund
    \begin{itemize}
      \item Jährliche Konferenz - "`State of the Map"', heuer: Buenos Aires
    \end{itemize}
      \pause
    \item Universitäten
    \begin{itemize}
      \item Bakk-, Master- und Doktorarbeiten mit OSM
      \item Server-Hosting
    \end{itemize}
      \pause
    \item Organisationen, die Daten sponsern
    \begin{itemize}
      \item Firmen wie Yahoo/Bing, die Luftbilder zur Verfügung stellen
      \item Regierungen mit besseren Open-Data-Gesetzen als Österreich
  % BSP TIGER, USA
  % Dänemark, Hausnummern
  % Frankreich,Tschechien: Kataster
    \end{itemize}
      \pause
    \item Firmen die mit OSM arbeiten, z.B.:
    \begin{itemize}
      \item Geofabrik (de)
      \item MapQuest (us)
      \item BikeCityGuide (Graz)
    \end{itemize}
  \end{itemize}



\end{frame}

  
{
 \usebackgroundtemplate{\includegraphics[height=10cm]{Osmdbstats2_users.png}}

\begin{frame}{Geschichte von OpenStreetMap}
  \vspace{0.6cm}
\begin{itemize}
  \item Start des Projekts im August 2004 durch \emph{Steve Coast}
  \item Dezember 2006 - Yahoo erlaubt abzeichnen
  \item Juli 2007 - Erste Konferenz, "`State Of The Map"'
  \item August 2007 - 10.000 Registrierte Benutzer
  \item März 2009 - 100.000 Registrierte Benutzer
  \item Januar 2010 - Haiti--Projekt
  \item November 2010 - Bing erlaubt abzeichnen
  \item Juli 2011 - Erste "`State Of The Map Europe"' in Wien
  \item Januar 2013 - 1.000.000 Registrierte Benutzer
  \item Gestern - 1.491.901 Registrierte Benutzer
\end{itemize}

\end{frame}
}




\begin{frame}{Warum OpenStreetMap?}

\hspace{0.5cm}Es beginnt 2004 mit einer Geschichte: 
  \vspace{0.3cm}

Ein Student ärgert sich, dass es in UK keine freien Geodaten gibt. 
  \vspace{0.3cm}

\parbox{9.5cm}{Die Daten auf streetmap.co.uk wurden mit Steuergeldern erstellt, man kann die Rohdaten jedoch nicht frei verwenden.}
\hfill
\raisebox{\dimexpr-\height+\baselineskip}{\includegraphics[height=1cm]{traurig.png}}

  \vspace{0.6cm}
\pause

Warum muss man für etwas, was bereits von der Allgemeinheit mit Steuergeld bezahlt wurde, nocheinmal bezahlen?
  \vspace{0.3cm}

\parbox{9.1cm}{Und darf es selbst dann nicht frei Nutzen? \\Doppelbesteuerung ist zumindest bei uns verboten?}
\hfill 
\raisebox{\dimexpr-\height+\baselineskip}{\includegraphics[height=1cm]{grantig.png}}

\pause

 \parbox{7.5cm}{\vspace{0.4cm}\hspace{0.5cm}$\Longrightarrow$ \hspace{0.5cm}Er gründet OpenStreetMap!} 
\raisebox{\dimexpr-\height+\baselineskip}{\includegraphics[height=1.2cm]{laugh.png}}

\end{frame}

\begin{frame}{Vision einer besseren Geo-Welt}

 Sollte es nicht so sein:
  \begin{itemize}
    \item Es gibt weltweit EIN Portal für ALLE Geodaten 
    \item In einheitlichem Format (WGS84, Dezimalgrad und UTF-8)
    \item Alte Versionen verfügbar sind (Um Änderungen zu tracken)
    \item Unkompliziert Fehler melden, oder selbst ausbessern kann
\pause
    \item Man keine Anträge stellen muss, sondern einfach einen Ausschnitt wählt und Rohdaten runterlädt
    \item Alle Daten unter einen freien Lizenz nutzen kann
  \end{itemize}

  \begin{columns}[c]
        \column{.5\textwidth}
        \begin{center}
  \includegraphics[width=3.5cm]{marble.png}
  \end{center}
        \column{.5\textwidth}
      \begin{center}
    \includegraphics[width=2.5cm]{cc-by-sa.pdf}
  \end{center}
\end{columns}

\end{frame}



\section{Wie funktioniert OpenStreetMap?}
% * Technogie, Datenmodell, Lizenz

\begin{frame}{Woher kommen unsere Daten?}

\begin{itemize}
  \item Freiwillige tragen ihr Wissen bei: Jeder weis viel über seine Umgebung:
	\begin{itemize}
	  \item Hausnummern, Straßennamen,
	  \item Restaurants, Bars, POIs, ...
  \end{itemize}
  \pause
  \item Bei Mapping-Parties werden \\ gezielt Gebiete verbessert... 
\end{itemize}

  \vspace{0.5cm}
 99\% Handarbeit!

  \vspace*{-3cm}
 \hfill \includegraphics[width=4.2cm]{alps_mp.jpg}


  \pause
\begin{itemize}
  \item Hin und wieder Importe aus Open Government Data
  \begin{itemize}
    \item USA, TIGER Data (2008)
    \item Dänemark, Hausnummern (laufend synchronisiert)
    \item Wien, Baumkataster
  \end{itemize}
\end{itemize}

\end{frame}

\subsection{Technologie}

% 100% freie Software
% → jeder kann den Software-Stack verwenden http://openaviationmap.org/
% Quality Assurance
% * Es gibt automatische Q/A-Tools
% * kaum Streitfälle - wenn dann Mailinglist, Data Working Group
% * 
% tolle Bilder herzeigen!
% * irgendein Zoo
% * 3D
% * Tolle Kartenstile:
%     * OSM-Fr?
%     * stamen watercolor
%     * pistemap
%     * bicycle map
%     * OpenSeaMap

\begin{frame}{Freie Software für freie Daten}

Nicht nur die Daten sind frei, sondern auch die Software um die Daten zu verwalten, generieren und verwenden.

\vspace{0.6cm}

Es wird zu 100\% Freie Software verwendet. Beispiel osm.org:
\begin{itemize}
  \item Hauptdatenbank: (PostgreSQL/PostGIS) 
  \item Renderstack: Mapnik/Tirex
  \item Webserver: Apache/mod\_tile
  \item Webfrontend: Leaflet
  \item Web-Backend: Basiert auf Ruby on Rails
  \item Doku-Portal: Mediawiki
\end{itemize}


  \vspace*{-3cm}
 \hfill \includegraphics[width=3cm]{tux.png}


\pause

Z.B verwendet \href{http://openaviationmap.org/}{openaviationmap.org} dieselben Technologien.

\end{frame}


\begin{frame}{Serverinfrastruktur}
Es gibt eine zentrale Datenbank (PostgreSQL/PostGIS) für Schreibzugriffe (in GB).\\
\pause
Diese wird weltweit gespiegelt für Lesezugriffe mit unterschiedlichen Methoden:

\begin{itemize}
  \item API-Lesezugriffe über mehrere Spiegel-Server lastverteilt
  \item Rendering-Server nutzen eine lokale, minütlich aktualisierte Datenbank
  \begin{itemize}
    \item Tileserver über GeoDNS weltweit verteilt (meist von Sponsoren)
  \end{itemize}
  \item Extrakte zum Download siehe \href{http://wiki.osm.org/Planet}{wiki.osm.org/Planet}
  \item Für räumliche SQL-Abfragen: Overpass API, zB alle italienischen Restaurants in Wien
\end{itemize}

\end{frame}


\begin{frame}{Versionierung}

Der komplette Datenbestand steht unter Versionskontrolle.
\begin{itemize}
  \item Auszüge können für beliebige Zeitpunkte erstellt werden
  \item Spiegel-DB mit inkrementellen diffs minütlich aktualisierbar
  \item DB sicher gegen Korrumption durch parallele Edits durch Verwendung von Changesets
  \begin{itemize}
    \item Pro Tag werden $\sim$16.500 Changesets submitted
  \end{itemize}
  \item Für jedes Objekt ist seine gesamte Historie abrufbar
\end{itemize}

% \vspace*{-3cm}
 \hfill \includegraphics[width=8cm]{history.png}


\end{frame}

\subsection{Datenmodell}

\begin{frame}{Datenmodell}
Wurde von Informatikern ohne Geo-Vorbelastung erstellt:
\begin{itemize}
  \item Punkte (Koordinaten), $\Rightarrow$ "`Node"' \includegraphics[width=0.5cm]{node.png}
  \begin{itemize}
    \item POIs
    \pause
    \item Teile von Wegen
  \end{itemize}
  \item Linienzüge sind eine Reihe von Nodes, $\Rightarrow$ "`Way"' \includegraphics[width=0.5cm]{way.png}
  \begin{itemize}
    \item Wege, Flüsse, Hecken etc.
    \item können geschlossen sein: Gebäude, Flächen
  \end{itemize}
  \pause
  \item Gruppierungen von Ways/Nodes $\Rightarrow$ "`Relations"' \includegraphics[width=0.5cm]{relation.png}
  \begin{itemize}
    \item Streckenrelationen, zB ÖPNV-Routen, Radrouten
    \item Multipolygone 
    \item Abbiegebeschränkungen (Way:von, Way:nach, Node:über)
    \item Meta-Relationen, zB für Verkehrsverbünde
  \end{itemize}
\end{itemize}

\end{frame}


\begin{frame}{Datenmodell: Tagging}

Jedes Element kann beliebige Anzahl Eigenschaften haben. 
Diese "`Tags"' genannten key=value Paare sind Freitext -- z.B.:
\begin{itemize}
  \item amenity = cafe \includegraphics[width=0.5cm]{cafe.png}
  \item highway = footway \includegraphics[width=1cm]{footway.png}
  \item building = yes  \includegraphics[width=0.5cm]{building.png}
\end{itemize}

\pause
Dadurch ist man zu 100\% flexibel - und kann praktisch alles in die OSM eintragen, was einen Geobezug hat!

\begin{itemize}
  \item Straßen- und Wegenetz, Schiffahrtsrouten, Skipisten, \dots
  \item Flächen (Bewuchs, Landnutzung, Schutzzonen)
  \item POI-Eigenschaften wie Kontaktdaten, Öffnungszeiten, Rollstuhleignung, Nichtraucherschutz
\end{itemize}

Standards werden im Wiki festgelegt, siehe \href{http://wiki.openstreetmap.org/wiki/DE:How\_to\_map\_a}{wiki/DE:How\_to\_map\_a}

\end{frame}

\begin{frame}{Datenmodell: Hintergrund}

\begin{itemize}
  \item Koordinaten sind in Dezimalgrad, WGS84.
  \begin{itemize}
  \item 3D-Information werden als Tags eingetragen, zb ele=435
\end{itemize}
  \item Genauigkeit 7 Stellen, entspricht 1\,cm am Äquator.
  \item Dateiformate: XML, PBF (Binärkomprimierung für Geodaten von Google)
\end{itemize}

\end{frame}





\begin{frame}{Qualitätssicherung}

\end{frame}



\begin{frame}{Lizenz}

Die Daten stehen unter der Open Database Licence - Entspricht etwa Creative Commons - Attribution - Sharealike für Daten.
\begin{itemize}
  \item Jeder darf die Daten, auch kommerziell verwenden
  \item Quelle: "`OpenStreetMap and Contributors, ODbL"' muß angegeben werden.
\end{itemize}

 \begin{center}
 \includegraphics[width=1cm]{ODbL.png}
 \hspace{2cm}
 \includegraphics[width=1.5cm]{cc-by-sa.png}
 \end{center}

\pause
Die Web-Karten auf \href{http://osm.org}{openstreetmap.org} sind CC-BY-SA.

\end{frame}

\section{Wie OpenStreetMap nutzen?}

\begin{frame}{Interaktive Web-Karten}

Hauptseite: \href{http://osm.org}{www.OpenStreetMap.org}

 \begin{center}
 \includegraphics[width=9cm]{mainpage.png}
 \end{center}



\end{frame}

\begin{frame}{Spezialkarten:}


  \begin{table}[htbp]
    \centering
    \begin{tabular}{r|l}
      Radkarte  &  \url{http://opencyclemap.org} \\
      Wanderkarte & \url{http://hikebikemap.de} \\
      Rauchfrei-Karte & \url{http://OpenGastroMap.org} \\
      Rollstuhl-Karte & \url{http://wheelmap.org} \\
      Seekarte & \url{http://OpenSeaMap.org} \\
\pause
      200 weitere: & siehe \href{http://wiki.openstreetmap.org/wiki/List\_of\_OSM\_based\_Services}{OSM-Wiki} \\
%      Interaktive Informationskarte & http://openstreetbrowser.org
    \end{tabular}
  \end{table}

  Blitzschnelles Routing: \url{http://osrm.at} \hspace{1cm} \includegraphics[width=2.5cm]{shoes.jpg}

\end{frame}

\begin{frame}{Mobil Nutzen}
	Apps:
 
 \begin{itemize}
   \item  Android ( \textgreater 70) \url{http://wiki.osm.org/Android}
   \item  iPhone ( \textgreater 60 )  \url{http://wiki.osm.org/Apple\_iOS}
   \item  Blackberry ( 8 ) \url{http://wiki.osm.org/BlackBerry\_OS}
 \end{itemize}
 
 \begin{center}
 \includegraphics[width=5cm]{tablet.jpg}
 \end{center}

 Natürlich auch auf Navis, am OSM-freundlichsten sind Garmin: \href{http://wiki.osm.org/Garmin}{wiki.osm.org/Garmin}!

\end{frame}

  \subsection{ OpenStreetMap Verbessern}

\begin{frame}{OpenStreetMap Verbessern}

  Eine große Auswahl an Editoren steht fürs Web, Desktop- und Mobilnutzung zur Verfügung

  \begin{itemize}
    \item Web:
    \begin{itemize}
	    \item Hauptseite - Edit: iD (JavaScript)
      \item JOSM web-start
      \item oder auch einfach nur Fehler melden!
	      \pause
    \end{itemize}
    \item Mobile (Auswahl): Alle siehe  \href{http://wiki.openstreetmap.org/wiki/Android\#OpenStreetMap\_editing\_features}{Android}, \href{http://wiki.openstreetmap.org/wiki/Apple\_iOS\#OpenStreetMap\_editing\_features}{iOS}:
    \begin{itemize}
      \item Vespucci: Ausgewachsener Editor
      \item osmaptuner: Existierende POIs ergänzen
      \item OsmTracker: GPS-Tracks, Audio, schnell POIs hinzufügen
    \end{itemize}
  \item Desktop
    \begin{itemize}
      \item \href{http://josm.openstreetmap.de}{JOSM}
      \item \href{http://merkaartor.be}{Merkaartor}
      \item ArcGIS (seit 10.1)
    \end{itemize}
  \end{itemize}

\end{frame}

\begin{frame}{Die Zukunft... 3D! }
  Neu! Jetzt auch in 3D! Beispielsweise auf \href{http://maps.osm2world.org/?zoom=17&lat=47.06156&lon=15.46983&layers=BF0FTFFF}{maps.osm2world.org}.

  \includegraphics[width=0.9\textwidth]{3d.png}


\end{frame}

\begin{frame}{Hilfe}
\begin{itemize}
  \item 
\end{itemize}

Fragen? 
\begin{itemize}
  \item Dokumentation: \href{http://wiki.openstreetmap.org}{wiki.openstreetmap.org}
  \item Immer noch etwas unklar? $\Rightarrow$ Mailingliste \href{http://lists.openstreetmap.org/listinfo/talk-at}{talk-at}
  \item Stammtische! Europaweit: \url{http://usergroups.openstreetmap.de/}
  \item Konferenz: \href{http://sotm-eu.org/}{State of the Map Europe}, 13.-15. Juni, Karlsruhe
\end{itemize}

  \pause
\end{frame}

\section{Ende}

\begin{frame}{Vielen Dank für die Aufmerksamkeit!}

  Folien zum Ageo Forum am 16.1.2014, Wien
\vspace{1cm}

Erstellt mittels \LaTeX Beamer, Quelltext: \href{https://github.com/species/vortrag-osm-ageo}{Github}.
\vspace{1cm}

\href{mailto:michael.maier@student.tugraz.at}{Michael Maier}

Twitter: \href{https://twitter.com/osmgraz}{@osmgraz}
\vspace{1cm}

Folien unter: \includegraphics[width=1cm]{cc-by-sa.pdf}. 

Alle Daten ODbL, OpenStreetMap Contributors.

\end{frame}

\end{document}
